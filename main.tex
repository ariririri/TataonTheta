\documentclass[uplatex,b5j,11pt]{jsbook}
\title{周期}

\usepackage[dvipdfmx]{graphicx}
\usepackage{fancybox}
\usepackage{amsmath,amssymb}
\usepackage{multicol}
\usepackage{ascmac}
\usepackage{booktabs}
\usepackage[dvipdfmx]{graphicx}
\usepackage[dvipdfmx]{color}
\usepackage{makeidx}
\usepackage{colortbl}
\usepackage{amsthm}
\usepackage{my-default}
\usepackage{tikz}
\usepackage{tikz-cd}

\begin{document}
\chapter{introduction}
\section{Definition of $\theta(z, \tau)$ and its preiodicity in z}
The central character in our theory is the analytic function $\theta(z, \tau)$ in 2 variables defined by
\begin{equation*}
    \theta(z, \tau) = \sum_{n \in \mathbb{Z}} \exp(i\pi n^2 \tau + 2\pi in z)
\end{equation*}
where $z \in \mathbb{C}, \tau \in \mathcal{H}$.

The series converges absolutely and uniformly on compact sets; in fact,
if $|\mathrm{Im} z| < c$ and $ \mathrm{Im} \tau > \epsilon$, then

\end{document}
