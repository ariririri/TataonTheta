\documentclass[uplatex,b5j,11pt]{jsbook}
\title{周期}

\usepackage[dvipdfmx]{graphicx}
\usepackage{fancybox}
\usepackage{amsmath,amssymb}
\usepackage{multicol}
\usepackage{ascmac}
\usepackage{booktabs}
\usepackage[dvipdfmx]{graphicx}
\usepackage[dvipdfmx]{color}
\usepackage{makeidx}
\usepackage{colortbl}
\usepackage{amsthm}
\usepackage{my-default}
\usepackage{tikz}
\usepackage{tikz-cd}

\begin{document}
\chapter{introduction}


\section{Definition of $\theta(z, \tau)$ and its preiodicity in z}
\label{Definition of theta(z, tau) and its preiodicity in z}
The central character in our theory is the analytic function $\theta(z, \tau)$ in 2 variables defined by
\begin{equation*}
    \theta(z, \tau) = \sum_{n \in \mathbb{Z}} \exp(i\pi n^2 \tau + 2\pi in z)
\end{equation*}
where $z \in \mathbb{C}, \tau \in \mathcal{H}$.

The series converges absolutely and uniformly on compact sets; in fact,
if $|\mathrm{Im} z| < c (\mathrm{Im}z > -c)$ and $ \mathrm{Im} \tau > \varepsilon$, then
\begin{align*}
|\exp(i\pi n^2 \tau + 2\pi in z)| & = \exp(- \pi n^2 \mathrm{Im}z) \cdot \exp(-2 \pi n \mathrm{Im}z) \\
                                  & \le \exp(-\pi n^2 \varepsilon) \cdot \exp( 2 \pi n c) \to 0
\end{align*}
if $n_0$ is chosen so that
\begin{equation*}
  \exp(- \pi \varepsilon n_0) \cdot \exp(-2 \pi c) < 1
\end{equation*}
then the inequaltiy
\begin{equation*}
     \exp(-\pi n^2 \varepsilon) \cdot \exp( 2 \pi n c)  \le \exp(- \pi \varepsilon (n^2 - nn_0))
\end{equation*}
shows that the series converges ant that too very rapidly.
(コンパクト空間上$z,t$のとり方によらず絶対収束する不等式で抑えた)

We may think of this series as the Fourier series for a function in z, periodic with respect to $z \mapsto z +1$
\begin{equation*}
 \theta (z, \tau) = \sum_{n \in \mathbb{Z}}a_n(\tau) \exp(2 \pi i n z), a_n(\tau) = \exp(\pi i n^2 \tau)
\end{equation*}
which displays the obvious fact that
\begin{equation*}
 \theta(z+1, \tau)  = \theta(z, \tau)
\end{equation*}

The precullar form

TBD

\section{$\theta (x, it)$ as tha fundamental periodic solution to the Heat equation}



\section{The Heisenberg group and theta functions with characteristics}
In addition to the standart theta functions discussed so far, there are variants called  \\
"\underline{theta functions with characteristics}" which play a very important role in understanding the functional equation and the identities satisfied by $\theta$, as well as the application of $\theta$ to elliptic curves.
These are best understood group-theoretically.
To explain this, let us fix $a, \tau$ and then rephrase the definition of the theta function $\theta(z,t)$ by introducing transformations as follows:

For every holomorphic function $f(z)$ and real numbers $a$ and $b$, let
\begin{align*}
(S_bf)(z) & = f(z+b) \\
(T_af)(z) & = \exp(i\pi a^2 \tau + 2\pi iaz) f(z+a \tau)
\end{align*}

Note then that:
\begin{align*}
  S_{b_1}(S_{b_2}f) & = S_{b_1 + b_2}(f) \\
  T_{a_1}(T_{a_2}f) & = \exp(i \pi a_1^2 \tau + 2\pi i a_1 z)(T_{a_2}f)(z+ a_1\ tau) \\
   &=\exp(i \pi a_1^2 \tau + 2\pi i a_1 z)\exp(i \pi a_2^2 \tau + 2\pi i a_2 (z +a_1 \tau))f(z + a_1 \tau + a_2 \tau) \\
   & = \exp(i\pi \tau (a_1^2 + a_2^2 + 2a_1a_2) + 2 \pi i (a_1 + a_2)z) f(z + a_1 \tau + a_2 \tau) = T_{a_1 + a_2}(f)
\end{align*}
These are the so called "\underline{1-parameter groups}", which means continuous homomorphism from $\mathbb{R}$ to groups.
Howeve, they do not commute!. We have:
\begin{align*}
  S_b(T_af)(z) &= (T_a)(f(z+b)) \\
              & = \exp(\pi a^2 i \tau + 2\pi ia(z+b)) f(z+a\tau + b)
\end{align*}
and
\begin{align*}
T_a(S_bf)(z)   & = \exp(\pi a^2 i \tau + 2\pi iaz) (S_bf)(z+a\tau) \\
   & = \exp(\pi a^2 i \tau + 2\pi iaz) f(z+a\tau + b)
\end{align*}


and hence
\begin{align}
S_b \circ T_a = \exp(2\pi i ab) T_a \circ S_b \tag{(*)}
\end{align}
The group of transformations generated by the $T_a$'s and $S_b$'s is the 3-dimensional group

\begin{equation*}
 \mathcal{G} = \mathbb{C}_1^{*} \times \mathbb{R} \times \mathbb{R}, (\mathbb{C}_1^* = \{z \in \mathbb{C} \mid |z| =1\}
\end{equation*}

where $(\lambda, a, b) \in \mathcal{G}$ stands for the transformation:
\begin{equation*}
 (U_{(\lambda, a, b)}f(z)) = \lambda (T_a \circ S_bf)(z)
\end{equation*}
This is because,
\begin{align*}
T_{a_1}S_{b_1}T_{a_2}S_{b_2} & = \exp(2\pi i a_2b_1) T_{a_1}T_{a_2}S_{b_1}S_{b_2} \\
                             & = \exp(2\pi a_2 b_1)T_{a_1+a_2}S_{b_1+b_2}
\end{align*}
hence,the group generated by $T,S$ is subset $\mathcal{G}$,
and
\begin{equation*}
U_{(\lambda, a,b)} = T_{a-1}S_{\lambda}T_1 S_{b- \lambda}
\end{equation*}

The group law on $\mathcal{G}$ is given by
\begin{equation*}
    (\lambda ,a ,b )(\lambda', a', b') = (\lambda \lambda' \exp(2\pi i ba'), a+a', b+b')
\end{equation*}

Note that
\begin{equation*}
    Z(\mathcal{G}) = \mathbb{C}_1^* =  [\mathcal{G}, \mathcal{G}]
\end{equation*}
特に証明がなかったがこれを証明しておくと,
$(\lambda, a, b) \in Z(\mathcal{G})$とすると,任意の$(\lambda', a', b') \in \mathcal{G}$に対し,
\begin{align*}
  (\lambda ,a ,b )(\lambda', a', b') & = (\lambda \lambda' \exp(2\pi i ba'), a+a', b+b') \\
                                     & =  (\lambda' ,a' ,b' )(\lambda, a, b) \\
                                     & = (\lambda \lambda' \exp(2\pi i b'a), a+a', b+b')
\end{align*}
となるので,$\exp(ab' - ba') =1$となり,$a =b =0$しかありえない.
第2,3成分は交換可能なので$[\mathcal{G}, \mathcal{G} \subset \mathbb{C}_1^*$であり,$a=a'=b=1,b'= n$として計算すればすべての$\mathbb{C}_1^*$が表せることがわかる.

この群は
\begin{equation*}
 \{1\} 	\vartriangleright  \mathbb{C}^*_{1} = [\mathcal{G}, \mathcal{G}]  \vartriangleright G
\end{equation*}
よりnilpotent groupとなる.
上で定めたnilpotent groupを\textbf{Heisenberg group}という.
In fact, the relation * is simly Weyl's integrated form of the Heisenberg commutation relations. Now recall that we have the \underline{classical theorem} of Von Neumann and Stone which says that $\mathcal{G}$ has a unique ireducible unitary representation in which $(\lambda, 0, 0)$ acts by $\lambda$ (identity)

ここではそのお気持ちが述べられているが、証明は不明.
$E$を$\mathbb{C}$上の正則関数全体とし,$f \in E$に対し,
\begin{equation*}
 ||f|| = \int_{\mathbb{C}} \exp(-2\pi y^2/ \mathrm{Im}\tau) |f(x+iy)|^{2}dxdy
\end{equation*}
と定め,
\begin{equation*}
\mathcal{H} =\{ f \in E \mid ||f|| < \infty\}
\end{equation*}
とする.真面目に積分計算するとUnitaryは示せる.
$\mathcal{H}$がHilbert spaceであることも示せる.
中線定理が成り立つことから内積空間であることがいえ,点列の極限を計算することで完備性が言える.
ただirreducibleかやuniqueは不明.ただこれはこれ以上扱わない.

To return to $\theta$; note that the subset
\begin{equation*}
 \Gamma = \{(1,a,b) \in \mathcal{G} \mid a,b \in \mathbb{Z}\}
\end{equation*}
is a subgroup of $\mathcal{G}$, By the characteriuzation of $\theta$ in \ref{Definition of theta(z, tau) and its preiodicity in z},
we see that, upto scalas, $\theta$ is the unique entire function invariant under $Gamma$. Suppose now that $\ell$ is a positive integer; set $\ell \Gamma = \{(1, \ell a, \ell b)\} \subset \Gamma$ and
\begin{equation*}
 V_{\ell} = \{\mbox{entire functions} f(z) \mbox{invariant under} \ell \Gamma\}
\end{equation*}
Then, we have the following

\begin{lem}
 An entire function $f(z)$ is in $V_{\ell}$ if and only if
 \begin{equation*}
  f(z) = \sum_{n \in 1/{\ell} \mathbb{Z}} c_n \exp(\pi i n^2 \tau + 2\pi i nz)
 \end{equation*}
 such that $c_n = c_m$ if $n-m \in \ell \mathbb{Z}$. In particular, $\mathrm{dim}\ell = \ell^2$.
\end{lem}
\begin{proof}
$f(z) \in V_{\ell}$の時,
$S_{\ell}$で不変なので,
\begin{equation*}
    S_{\ell}(f)(z) = f(\ell + z) = f(z)
\end{equation*}
となるので周期$\ell$を持ち、それについてFourier Expansionできる.
\begin{align*}
    f(z) = \sum_{n \in 1/\ell \mathbb{Z}} c_n' \exp (2\pi in z)
\end{align*}
$c_n' = c_n \exp(\pi i n^2 \tau)$としして$f(z)$に$T_{\ell}$ を作用させると,
\begin{align*}
    T_{\ell}(f)(z) & = f(z + \ell \tau ) \exp( \pi i \ell^2\tau + 2\pi i \ell z) \\
                   & = \sum c_n \exp(\pi i n^2\tau  + 2 \pi in (z+\ell \tau) \exp( \pi i \ell^2\tau + 2\pi i \ell z)
                   & = \sum c_n \exp(\pi i (n + \ell)^2\tau  + 2 \pi in (z+\ell))
\end{align*}
より$c_n = c_{n = \ell}$となる.逆は明らか
\end{proof}

For $m \in \mathbb{N}$. let $\mu_m \subset \mathbb{C}_1^*$ be the group of m-th roots of 1. For $\ell \in \mathbb{N}$, let $\mathcal{G}_{\ell}$ be the finite group defined as
\begin{equation*}
 \widetilde{\mathcal{G}_{\ell}} = \mu_{\ell^2} \times 1/\ell \mathbb{Z} \times 1/\ell \mathbb{Z}
\end{equation*}
with group law given by
\begin{equation*}
    (\lambda ,a ,b )(\lambda', a', b') = (\lambda \lambda' \exp(2\pi i ba'), a+a', b+b')
\end{equation*}
この時,$\ell \Gamma \subset \widetilde{\mathcal{G}_{\ell}}$は正規部分群になるので,それで割ることができ,
\begin{equation*}
    \mathcal{G}_{\ell} := \mu_{\ell^2} \times \frac{1}{\ell}\mathbb{Z}/ \ell \mathbb{Z} \times \frac{1}{\ell}\mathbb{Z}/ \ell \mathbb{Z}
\end{equation*}
となる.
Noe the elements $S_{1/ \ell}, T_{1/ \ell} \in \mathcal{G}$ commute with $\ell \Gamma$(in view of *) and hence act on $V_{\ell}$. This goes down to an action of $\mathcal{G}_{\ell}$ on $V_{\ell}$; in fact, exactly like $\mathcal{G}$, the generators $S_{1/ \ell}$ of $\mathcal{G}_{\ell}$ act on $V_{\ell}$ as follows:
\begin{equation*}
 S_{1/\ell}(\sum c_n \exp (\pi in^2 \tau + 2\pi in z)) = \sum c_n \exp(2\pi in / \ell)\exp( \pi i n^2 \tau + 2 \pi in z)
\end{equation*}
and
\begin{align*}
 T_{1/ \ell}f(z) & = f(z + 1/ \ell \tau)\exp(\pi i 1/\ell^2 \tau + 2\pi)
\end{align*}


\begin{lem}
 $\mathcal{G}_{\ell}$ acts irreducibily on $V_{\ell}$.
\end{lem}

TBD

\section{Projetive Embedding of $\mathcal{C}/ \mathbb{Z} + \mathbb{Z}\tau$ by means of theta functions}

The theta functions $\theta_{a,b}$ defined above have avery important geometric application. Take any $\ell \ge 2$. Let $E_{\tau}$ be the complex torus $\mathbb{C}/\Lambda_{\tau}$ where $\Lambda_{\tau} = \mathbb{Z} + \mathbb{Z}\tau$.
Let $(a_i, b_i)$ be a set of coset representaiton of $( \frac{1}{\ell} \mathbb{Z}/ \mathbb{Z})^2$, $0 \le i \le \ell^2 - 1$.
Write $\theta_i = \theta_{a_i, b_i}$.
For all $z \in \mathbb{C}$, consider the $\ell^2$-tuple
\begin{equation*}
  (\theta_0(\ell z, \tau), \ldots, \theta_{\ell^2 -1}(\ell z, \tau))
\end{equation*}

これは$\mathbb{P}_{\mathbb{C}}^{\ell^2-1}$の元を定める.さらに実際は$\phi_{\ell}:E_{\tau} \to \mathbb{P}^{\ell^2 -1}$を定める.

このWell-defined性は
\begin{itemize}
    \item 上の写像が定義域の代表元のとり方によらない.つまり$a \in \Lambda_{\tau}$を足した時に定数倍
 $(\theta_0(\ell z, \tau), \ldots, \theta_{\ell^2 -1}(\ell z, \tau)) = \lambda(\theta_0(\ell z + a, \tau), \ldots, \theta_{\ell^2 -1}(\ell z + a, \tau))$
 になっている。(実際は生成元の場合だけ示せばよい)
 \item 値域が射影空間の外に出ない.つまり,全て同時に0とならない.
\end{itemize}
まず最初の代表元のとり方によらないことを見ておく.
$\theta_{a,b}(z+ \ell, \tau) = \sum_{n \in \mathbb{Z}} \exp (\pi i (a+n)^2\tau + 2\pi i(n+a)(z+\ell+b))$であり,$a \ell \in \mathbb{Z}$より,
$\theta_{a,b}(z, \tau)$と一致する.
\begin{align*}
\theta_{a,b}(z+ \ell \tau, \tau) & = \sum_{n \in \mathbb{Z}} \exp (\pi i (a+n)^2\tau + 2\pi i(n+a)(z+\ell \tau+b))  & \mbox{by definition}\\
                                 & = \sum_{n \in \mathbb{Z}} \exp (\pi i (a^2+n^2 + 2an + 2n \ell + 2 a \ell)\tau + 2\pi i(n+a)(z \tau+b)) \\
                                 & = \sum_{n \in \mathbb{Z}} \exp (\pi i ((a+n +\ell)^2 - \ell^2)\tau + 2\pi i(n+a)(z \tau+b)) \\
                                 & = \sum_{n \in \mathbb{Z}} \exp (\pi i ((a+n +\ell)^2 - \ell^2)\tau + 2\pi i\left((n+a + \ell)(z \tau+b)  - \ell (z \tau + b)\right) \\
                                 & = \sum_{n \in \mathbb{Z}} \exp (\pi i ((a+n +\ell)^2 - \ell^2)\tau + 2\pi i\left((n+a + \ell)(z \tau+b)  - \ell z \tau \right) \\
                                 & = \sum_{n \in \mathbb{Z}} \exp(- \pi i \ell^2 - 2\pi i \ell z \tau)\exp (\pi i ((a+n + \ell)^2 \tau + 2\pi i (n+a + \ell)(z \tau+b)  ) \\
                                 & = \lambda \sum_{n \in \mathbb{Z}} \exp(- \pi i \ell^2 - 2\pi i \ell z \tau)\exp (\pi i ((a+n + \ell)^2 \tau + 2\pi i (n+a + \ell)(z \tau+b)  ) \\
                                 & = \lambda \sum_{n \in \mathbb{Z}} \exp(- \pi i \ell^2 - 2\pi i \ell z \tau)\exp (\pi i ((a+n )^2 \tau + 2\pi i (n+a )(z \tau+b)  )  & n = n+ \ell \mbox{の置き換え}\\
\end{align*}
ただし,$\exp(- \pi i \ell^2 - 2\pi i \ell z \tau = \lambda$とおいた.
これより$a,b$のとり方によらず一定変化するため,well-definedである.

全て0にならないことは以下からわかる.

\begin{lem}
0でない$f \in V_{\ell}$は$\mathbb{C}/ \ell \Lambda_{\tau}$の基本領域上に$\ell^2$個のゼロ点を持つ.
$\theta_{a,b}$の場合,それは$(a+p+\frac{1}{2},b+q+ \frac{1}{2})(p, q \in \mathbb{Z})$となる.
ここから特に$i \neq j$の時,$\theta_i, \theta_j$のゼロ点は異なる.
\end{lem}


\begin{proof}
$f$の零点は軸上にないように必要ならば平行移動して考える.(0でない正則関数が有界区間上で0となる個数は有限個なので)
この時,ゼロ点の個数は

\begin{equation*}
    \mbox{number of zeros of f} = \frac{1}{2 \pi i} \int_{\sigma + \sigma^* + \delta + \delta^*} \frac{f'}{f}dz
\end{equation*}
となる.

\tikzset{every picture/.style={line width=0.75pt}} %set default line width to 0.75pt

\begin{tikzpicture}[x=0.75pt,y=0.75pt,yscale=-1,xscale=1]
%uncomment if require: \path (0,235); %set diagram left start at 0, and has height of 235

%Shape: Parallelogram [id:dp6661801141741219]
\draw   [](160.4,71) -- (292,71) -- (235.6,175) -- (104,175) -- (160.4, 71) ;
\draw (220,70) node [anchor=north west][inner sep=0.75pt]   [align=left] {$\sigma^{*}$};
\draw (200,174) node [anchor=north west][inner sep=0.75pt]   [align=left] {$\sigma$};
\draw (250,115) node [anchor=north west][inner sep=0.75pt]   [align=left] {$\delta$};
\draw (118,115) node [anchor=north west][inner sep=0.75pt]   [align=left] {$\delta^*$};

%Shape: Parallelogram [id:dp9052013569123434]
\draw  [dash pattern={on 4.5pt off 4.5pt}] (187.4,52) -- (319,52) -- (262.6,156) -- (131,156) -- cycle ;

%Straight Lines [id:da9267531477244373]
\draw    (131,156) -- (470,156) ;
\draw [shift={(472,156)}, rotate = 180] [color={rgb, 255:red, 0; green, 0; blue, 0 }  ][line width=0.75]    (10.93,-3.29) .. controls (6.95,-1.4) and (3.31,-0.3) .. (0,0) .. controls (3.31,0.3) and (6.95,1.4) .. (10.93,3.29)   ;
%Straight Lines [id:da06384579325368311]
\draw    (131,156) -- (131,18) ;
\draw [shift={(131,16)}, rotate = 450] [color={rgb, 255:red, 0; green, 0; blue, 0 }  ][line width=0.75]    (10.93,-3.29) .. controls (6.95,-1.4) and (3.31,-0.3) .. (0,0) .. controls (3.31,0.3) and (6.95,1.4) .. (10.93,3.29)   ;

% Text Node
\draw (122,143) node [anchor=north west][inner sep=0.75pt]   [align=left] {0};
% Text Node
\draw (172,30) node [anchor=north west][inner sep=0.75pt]   [align=left] {$\displaystyle \ell  \tau $};
% Text Node
\draw (316,29) node [anchor=north west][inner sep=0.75pt]   [align=left] {$\displaystyle \ell  \tau  + \ell $};
% Text Node
\draw (264.6,159) node [anchor=north west][inner sep=0.75pt]   [align=left] {$\displaystyle \ell  $};

\end{tikzpicture}

$f(z+ \ell) = f(z)$より$\int_{\sigma  + \sigma^*}f = 0$となる.
$f(z + \ell \tau) = const \exp(-2 \pi i \ell z) f(z)$となるので,$f(z + \ell \tau)' = const ( -2\pi i \ell \exp(-2 \pi i \ell z)f(z) + \exp(-2 \pi i \ell z)f'(z))$となり.
\begin{align*}
   \int_{\delta  + \delta^*}\frac{f'}{f}& = \int_{\delta} \frac{f'(z)}{f(z)}dz + \int_{- \delta} \frac{f'(z+ \ell \tau)}{f(z + \ell \tau)}dz \\
                                                & = \int_{\delta} \frac{f'(z)}{f(z)} - \frac{ const ( -2\pi i \ell \exp(-2 \pi i \ell z)f(z) + \exp(-2 \pi i \ell z)f'(z))}{ const \exp(-2 \pi i \ell z) f(z)} \\
                                                & = \int_{\delta} \frac{f'(z)}{f(z)} - \frac{ ( -2\pi i \ell f(z) + f'(z))}{ f(z)} \\
                                                & = \int_{\delta}  - \frac{  -2\pi i \ell f(z) }{ f(z)} \\
                                                & = 2 \pi i \ell^2
\end{align*}
よって$\ell^2$個であることが言えた.
$\theta(z, \tau)$は$\mathbb{C}/ \Lambda_{\tau}$上一つだけゼロ点を持つ.($\ell = 1$).

\begin{align*}
    \theta_{1/2, 1/2}(-z, \tau) & = \sum_{n \in \mathbb{Z}} \exp \left(\pi i (n+ 1/2)^2\tau + 2 \pi i (n+ 1/2)(-z + 1/2)\right) \\
                                & = \sum_{m \in \mathbb{Z}} \exp \left(\pi i (-m - 1/2)^2\tau + 2 \pi i (-m - 1/2)(-z + 1/2)\right)  & m = -n-1\\
                                & = \sum_{m \in \mathbb{Z}} \exp \left(\pi i (m + 1/2)^2\tau + 2 \pi i (m + 1/2)(z + 1/2)  - 2\pi i (m+1/2)\right)\\
                                & = - \sum_{m \in \mathbb{Z}} \exp \left(\pi i (m + 1/2)^2\tau + 2 \pi i (m + 1/2)(z + 1/2)\right)\\
                                & = - \theta_{1/2, 1/2}(z, \tau)
\end{align*}

これから$\theta_{1/2, 1/2}$は$z=0$でゼロになる.(これは具体的に計算して0を示すのは難しいんだろうか)
後は周期性から計算すればわかる.
\end{proof}
この後$\phi_{\ell}$の群作用やEmbeddingになっていることを見る.(TBD)

\section{Riemann's theta relations}

記号を改めてここで定義する.

$\theta(z, \tau):= \sum_{n \in \mathbb{Z}}  \exp (i \pi n^2 \tau + 2i \pi z)$,
$\Lambda_{\tau}:= \mathbb{Z}+ \mathbb{Z}\tau$,
$S_a\theta(z, \tau) = \theta(z + a, \tau)$,
$T_a\theta(z, \tau) = \exp(\pi i a^2 \tau + 2\pi i a z) \theta(z + a\tau, \tau)$

$A$を$A^tA = m^2I_n$となる行列とする.例えば

\begin{equation*}
A =
\begin{pmatrix}
1 &  1 &  1 &  1 \\
1 &  1 & -1 & -1 \\
1 & -1 &  1 & -1 \\
1 & -1 & -1 &  1 \\
\end{pmatrix}
\end{equation*}
とすると$A^tA = 4I_4$となる.これは
\begin{equation*}
 (x+y+u+v)^2 + (x+y-u-v) + (x-y+u-v)^2 + (x-y-u+v)^2 = 4(x^2 + y^2 + u^2 + v^2)
\end{equation*}
を定める.式が複雑になるため$\theta(z):=\theta(z, \tau)$, $\Lambda := \Lambda_{\tau}$と定める.

これらを使っていくつか式を算出する.
\begin{equation*}
B(0) := \theta(x)\theta(y)\theta(u)\theta(v) = \sum_{n,m,p,q \in \mathbb{Z}} \exp\left( \pi i (\sum n^2) \tau + 2 \pi i (\sum xn)\right)
\end{equation*}
ただし,$\sum xn = xn + ym + up + vq$であり,$x,n$を走る和は同様に表記する.

\begin{align*}
B\left(\frac{1}{2}\right) := &\theta(x+\frac{1}{2})\theta(y+\frac{1}{2})\theta(u+\frac{1}{2})\theta(v+\frac{1}{2}) \\
 = &\sum_{n,m,p,q \in \mathbb{Z}} \exp\left( \pi i (\sum n^2) \tau + 2 \pi i (\sum xn) +  \pi i \sum n\right)  & \mbox{ただx+1/2を展開しただけ}
\end{align*}

\begin{align*}
\exp( \pi i n^2 \tau + 2 \pi i n(x+\frac{1}{2}\tau)) & = \exp(\pi i \tau(n+\frac{1}{2})^2 - \frac{1}{4} \pi i \tau  + 2\pi i (n+ \frac{1}{2})x - \pi i x  & \mbox{平方完成}\\
\exp(1/4 \pi i \tau + \pi i x +  \pi i n^2 \tau + 2 \pi i n(x+\frac{1}{2}\tau)) & = \exp(\pi i \tau(n+\frac{1}{2})^2   + 2\pi i (n+ \frac{1}{2})x & \mbox{負の項を移項}
\end{align*}
なので,これの和を取ると,

\begin{align*}
B\left(\frac{1}{2}\tau\right) := & \exp\left(\pi i (\tau + \sum x)\right)\theta(x+\frac{1}{2})\theta(y+\frac{1}{2}\tau)\theta(u+\frac{1}{2}\tau)\theta(v+\frac{1}{2}\tau) \\
 = &\sum_{n,m,p,q \in \mathbb{Z}} \exp\left( \pi i \left(\sum (n+1/2)^2 \right) \tau + 2 \pi i \left(\sum x(n+ 1/2) \right) \right)
\end{align*}

\begin{align*}
\exp( \pi i n^2 \tau + 2 \pi i n(x+\frac{1}{2}+\frac{1}{2}\tau)) & = \exp(\pi i \tau(n+\frac{1}{2})^2 - \frac{1}{4} \pi i \tau  + 2\pi i (n+ \frac{1}{2})x - \pi i x + \pi in & \mbox{平方完成}\\
\exp(1/4 \pi i \tau + \pi i x +  \pi i n^2 \tau + 2 \pi i n(x+ \frac{1}{2}+\frac{1}{2}\tau)) & = \exp(\pi i \tau(n+ \frac{1}{2})^2   + 2\pi i (n+ \frac{1}{2})x  + \pi in) & \mbox{負の項を移項}
\end{align*}

\begin{align*}
B\left( \frac{1}{2}+\frac{1}{2}\tau\right) := & \exp\left(\pi i (\tau + \sum x)\right)\theta(x+\frac{1}{2} + \frac{1}{2}\tau)\theta(y+\frac{1}{2}+\frac{1}{2}\tau)\theta(u+\frac{1}{2}+\frac{1}{2}\tau)\theta(v+\frac{1}{2}+\frac{1}{2}\tau) \\
 = &\sum_{n,m,p,q \in \mathbb{Z}} \exp\left( \pi i \left(\sum (n+1/2)^2 \right) \tau + 2 \pi i \left(\sum x(n+ 1/2) \right) +  \pi i \sum n \right)
\end{align*}

$B(1/2) = \exp(\pi i \sum n)B(0), B(1/2+1/2 \tau) =  \exp(\pi i \sum n)B(1/2)$になること
と
\begin{equation*}
B(0) + B(1/2\tau) = \sum_{n,m,p,q \in 1/2 \mathbb{Z}}  \exp\left( \pi i \left(\sum n^2 \right) \tau + 2 \pi i \left(\sum xn \right) \right)
\end{equation*}
となり,
$\exp(\pi i \sum n)$は$\sum$が偶数のときは2倍になり,奇数の場合は消えるので、

\begin{align*}
\sum_{\eta = 0, 1/2, 1/2\tau 1/2 + 1/2\tau} B(\eta) = 2 \sum_{n,m,p,q \in 1/2 \mathbb{Z}}  \exp\left( \pi i \left(\sum n^2 \right) \tau + 2 \pi i \left(\sum xn \right) \right)
\end{align*}
ただし,$n,m,p,q$は全て整数か全て$1/2 + \mathbb{Z}$の元であり
さらに合計が偶数になるところを走る.

\begin{align*}
    n_1 & = \frac{1}{2}(n+m+p+q) & x_1 & = \frac{1}{2}(x+y+u+v) \\
    m_1 & = \frac{1}{2}(n+m-p-q) & y_1 & = \frac{1}{2}(x+y-u-v) \\
    p_1 & = \frac{1}{2}(n-m+p-q) & u_1 & = \frac{1}{2}(x-y-u+v) \\
    q_1 & = \frac{1}{2}(n-m-p+q) & v_1 & = \frac{1}{2}(x-y-u+v) \\
\end{align*}
とすると,
$\sum n^2 = \sum n_1^2$,$\sum xn = \sum x_1 n_1$となるので,
\begin{align*}
\sum_{\eta = 0, 1/2, 1/2\tau 1/2 + 1/2\tau} B(\eta) & = 2 \sum_{n,m,p,q \in 1/2 \mathbb{Z}}  \exp\left( \pi i \left(\sum n^2 \right) \tau + 2 \pi i \left(\sum xn \right) \right) \\
 & = 2 \sum_{n_1, m_1, p_1, q_1} \exp(\pi i \sum n_1^2) \tau + 2 \pi i (\sum x_1n_1))
\end{align*}


これより,以下の関係式が得られる.

\begin{equation*}
(R_1):
\sum_{\eta = 0, 1/2, 1/2\tau 1/2 + 1/2\tau} e_{\eta}\theta(x+\eta)\theta(y+\eta)\theta(z+\eta)\theta(v+\eta)  = 2 \sum_{n,m,p,q \in 1/2 \mathbb{Z}}  \exp\left( \pi i \left(\sum n^2 \right) \tau + 2 \pi i \left(\sum xn \right) \right) \\
\end{equation*}


これを$\theta_{a,b}$を用いて表す.
$\theta_{a,b} = T_aS_b\theta = \exp(\pi i a^2\tau + 2\pi i a(z+b))\theta(z+a\tau + b, \tau)$なので,
\begin{align*}
\theta_{0,0} & = \theta(z, \tau) \\
\theta_{0,\frac{1}{2}} &= \theta(z + \frac{1}{2}, \tau) \\
\theta_{\frac{1}{2},0} &= \exp(\pi i \frac{1}{4} + \pi i z )\theta(z+1/2\tau, \tau)) \\
\theta_{\frac{1}{2}, \frac{1}{2}} &= \exp(\pi i \tau/4 + \pi i (z + \frac{1}{2}))\theta(z + \frac{1}{2}(1 + \tau), \tau)
\end{align*}
である.これらを$\theta_{0,0}, \theta_{0,1}, \theta_{1,0}, \theta_{1,1}$と表す.

これらには以下の関係がある.
\begin{align*}
    \theta_{1, 1}(-z, \tau) & = \sum_{n \in \mathbb{Z}} \exp \left(\pi i (n+ 1/2)^2\tau + 2 \pi i (n+ 1/2)(-z + 1/2)\right) \\
                                & = \sum_{m \in \mathbb{Z}} \exp \left(\pi i (-m - 1/2)^2\tau + 2 \pi i (-m - 1/2)(-z + 1/2)\right)  & m = -n-1\\
                                & = \sum_{m \in \mathbb{Z}} \exp \left(\pi i (m + 1/2)^2\tau + 2 \pi i (m + 1/2)(z + 1/2)  - 2\pi i (m+1/2)\right)\\
                                & = - \sum_{m \in \mathbb{Z}} \exp \left(\pi i (m + 1/2)^2\tau + 2 \pi i (m + 1/2)(z + 1/2)\right)\\
                                & = - \theta_{1, 1}(z, \tau)
\end{align*}
また,ほかは以下となる.
\begin{align*}
\theta_{0,0}(-z ,\tau)  &= \sum \exp(\pi i (-n)^2 \tau + 2 \pi i (-n)z)) = \theta_{0,0}(z, \tau) \\
\theta_{0, 1}(-z, \tau) &= \sum \exp(\pi i (-n)^2 \tau + 2\pi i n(-z + \frac{1}{2}) ) = \sum \exp(\pi i (-n)^2 \tau + 2\pi i (-n)(z + 1/2))   + 2n \pi i ) = \theta_{0, 1}(z, \tau) \\
\theta_{1, 0}(-z, \tau) &= \sum \exp( \pi i (n+ \frac{1}{2})^2 \tau  - 2 \pi i(n + \frac{1}{2})z ) = \theta_{1, 0}(z, \tau)  \mbox{n=-n-1}
\end{align*}

これを使うと以下の関係式が得られる.

\begin{equation*}
 (R_2): \sum \theta_{i,j}(x)\theta_{i,j}(y)\theta_{i,j}(u)\theta_{i,j}(v) = 2\theta_{0,0}(x_1)\theta_{0,0}(y_1)\theta_{0,0}(u_1)\theta_{0,0}(v_1)
\end{equation*}
$x$を$x+1$に置き換えると,
$\sum \exp(\pi i n^2 \tau + 2 \pi i n z  + 2\pi in) = \sum \exp(\pi i n^2 \tau + 2 \pi i n z)$,
$\exp(\pi i \frac{1}{4} + \pi i (z+1) ) = - \exp(\pi i \frac{1}{4} + \pi i z)$となるので,
\begin{align*}
 (R_3): & \   \theta_{0,0}(x)\theta_{0,0}(y)\theta_{0,0}(u)\theta_{0,0}(v) + \theta_{0,1}(x)\theta_{0,1}(y)\theta_{0,1}(u)\theta_{0,1}(v) \\
  &- \theta_{1,0}(x)\theta_{1,0}(y)\theta_{1,0}(u)\theta_{1,0}(v) - \theta_{1,1}(x)\theta_{1,1}(y)\theta_{1,1}(u)\theta_{1,1}(v) \\
  &= 2\theta_{0,1}(x_1)\theta_{0,1}(y_1)\theta_{0,1}(u_1)\theta_{0,1}(v_1)
\end{align*}

となる.同様に$x = x + \tau$とすると,
\begin{align*}
 \exp(\pi i \tau + 2\pi ix) \theta_{0,0}(z + \tau ,\tau)  &= \sum \exp(\pi i \tau + 2\pi ix) \exp(\pi i (n)^2 \tau + 2 \pi i (n)(x + \tau)) = \theta_{0,0}(x, \tau) \\
 & = \sum \exp(\pi i (n+1)^2 + 2\pi (n+1)x) = \theta_{0,0}(x, \tau) \\
\exp(\pi i \tau + 2\pi ix) \theta_{0, 1}(x + \tau, \tau) &= \sum \exp(\pi i (n+ 1)^2 \tau + 2\pi i n(x  + \frac{1}{2} + \tau ) \\
& = \sum \exp(\pi i (n+1)^2 \tau + 2\pi i (n+1)(x + 1/2))   - \pi i ) = - \theta_{0, 1}(z, \tau) \\
\exp(\pi i \tau + 2\pi ix) \theta_{1, 0}(x + \tau, \tau) &= \sum \exp(\pi i \tau + 2\pi ix)\exp(\pi i \tau/4 + \pi i (x+\tau))\exp(\pi i n^2 \tau + 2\pi i n(x  + \frac{3}{2}\tau ) \\
      & = \sum \exp (\pi i \tau + 2\pi ix +\pi i \tau/4 + \pi i (x+\tau)   \\
      & + \pi i \tau (n+1)^2- \pi i \tau + 2\pi i (n+1)(x+ \frac{1}{2}\tau ) -  2\pi i (x + \frac{1}{2}\tau)) \\
      & = \sum \exp (\pi i \tau /4 + \pi ix)\exp(\pi\tau (n+1)^2 + 2\pi i(n+1)(x+\frac{1}{2}\tau) = \theta_{1,0}(x) \\
\exp(\pi i \tau + 2\pi ix) \theta_{1, 1}(x + \tau, \tau) & =  - \theta_{1,1}(x)
\end{align*}
となる.

$2x = x_1 + y_1 + u_1 + v_1$なので,
$\exp(\pi i \tau + 2\pi ix) = \exp \left(\sum ( \pi i \tau / 4 + \pi i x_1)\right)$となる.
よって
\begin{align*}
 (R_4): & \   \theta_{0,0}(x)\theta_{0,0}(y)\theta_{0,0}(u)\theta_{0,0}(v) - \theta_{0,1}(x)\theta_{0,1}(y)\theta_{0,1}(u)\theta_{0,1}(v) \\
  &+ \theta_{1,0}(x)\theta_{1,0}(y)\theta_{1,0}(u)\theta_{1,0}(v) - \theta_{1,1}(x)\theta_{1,1}(y)\theta_{1,1}(u)\theta_{1,1}(v) \\
   = & 2\theta_{0,1}(x_1)\theta_{0,1}(y_1) \theta_{0,1}(u_1) \theta_{0,1}(v_1)
\end{align*}
となる.

同様に
$x$を$x+ \tau +1$に置き換えることで,
\begin{align*}
 (R_5): & \   \theta_{0,0}(x)\theta_{0,0}(y)\theta_{0,0}(u)\theta_{0,0}(v) - \theta_{0,1}(x)\theta_{0,1}(y)\theta_{0,1}(u)\theta_{0,1}(v) \\
  &- \theta_{1,0}(x)\theta_{1,0}(y)\theta_{1,0}(u)\theta_{1,0}(v) + \theta_{1,1}(x)\theta_{1,1}(y)\theta_{1,1}(u)\theta_{1,1}(v) \\
   = & 2\theta_{1,1}(x_1)\theta_{1,1}(y_1) \theta_{1,1}(u_1) \theta_{1,1}(v_1)
\end{align*}
が得られる.

また$x$を$x+1/2, x+1/2\tau$とすることで同様に様々な公式が得られる.


またこうして得られた式から
$x=y,u=v$とすると,$x_1=x+v, y_1 = x-v, u_1 = 0, v_1=0$となる.
$\theta_{1,1}(0) = 0$より.$R_5$の右辺は0になる.
これから等式を変形すると
\begin{equation*}
    \theta_{0,0}(x)^2 \theta_{0,0}(u)^2 + \theta_{1,1}(x)^2 \theta_{1,1}(u)^2 = \theta_{0,1}(x)^2 \theta_{0,1}(u)^2 + \theta_{0,1}(x)^2 \theta_{0,1}(u)^2
\end{equation*}
となり,
また$R2+R5$より,
\begin{equation*}
    \theta_{0,0}(x)^2 \theta_{0,0}(u)^2 + \theta_{1,1}(x)^2 \theta_{1,1}(u)^2 = \theta_{0,1}(x)^2 \theta_{0,1}(u)^2 + \theta_{1,0}(x)^2 \theta_{1,0}(u)^2
    = \theta_{00}(x+u)\theta_{00}(x-u) \theta_{00}(0)^2
\end{equation*}
となる.
同様に$x,u$に対して$x+u,x-u$に関する 関係式が得られる.

こうした関係式に$u=0$を代入することで,
\begin{equation*}
    \theta_{0,1}(x)^2 \theta_{0,1}(0)^2 + \theta_{1,0}(x)^2 \theta_{1,0}(0)^2
    = \theta_{00}(x)^2 \theta_{00}(0)^2
\end{equation*}
が得られる.同様に(いろいろ計算すると)
\begin{equation*}
    \theta_{0,1}(x)^2 \theta_{1,0}(0)^2 - \theta_{1,0}(x)^2 \theta_{0,1}(0)^2
    = \theta_{11}(x)^2 \theta_{00}(0)^2
\end{equation*}
が得られる.

上の関係式に$x=0$を代入すると
\begin{equation*}
    \theta_{0,1}(0)^4  + \theta_{1,0}(0)^4
    = \theta_{00}(0)^4
\end{equation*}

が得られ,これは \textbf{ヤコビの恒等式}と呼ばれる.


\section{Doubly periodic meromorphic functions via $\theta(z, \tau)$}
この章では4つの手段で$E_{\tau}$上のmeromorphic functionを作る.
これは$\mathbb{C}$上の有理型関数であって,$\Lambda_{\tau}$上周期的であればよい.

\subsection{By restriction of rational functions from $\mathcal{P}^3$}
$\phi_{\ell}: \mathbb{C}/\Lambda_{\tau} \to \mathbb{P}^{\ell^2-1}, (\theta_0(\ell z, \tau), \ldots, \theta_{\ell^2-1}(\ell z, \tau)$はembeddingだったので,
$\ell = 2$の場合にもembeddingになっている。 そこで,$\mathbb{C} \to \mathbb{P}^{1}, z \mapsto \frac{\theta_{a,b}}{\theta_{0,0}}(a,b \in \{0, 1\}$)はmeromorphicになる.

\subsection{As quotients of products of translates of $\theta(z)$ itself}

$a_1, \ldots, a_k, b_1, \ldots, b_k$を$\sum a_i = \sum b_i$とする.
この時
\begin{equation*}
 \prod_{1 \le i \le k} \frac{\theta(z -a_i)}{\theta(z - b_i)}
\end{equation*}
は$\Lambda_{\tau}$上周期的である.それは,
\begin{align*}
    \theta(z + 1) & = \sum \exp(\pi i n^2 \tau + 2 \pi i n (z+1)) \\
                  & = \sum \exp(\pi i n^2 \tau + 2 \pi i n (z) \\
                  & = \theta(z) \\
    \theta(z + \tau) & = \sum \exp(\pi i n^2 \tau + 2 \pi i n (z+\tau)) \\
                      & = \sum \exp(\pi i (n+1)^2 \tau  - \pi i \tau + 2 (n+1)z \pi i - 2 z\pi i ) \\
                      & = \sum \exp(-\pi i \tau - 2\pi i z) \theta(z)
\end{align*}
より,$\theta(z- a+1) = \theta(z-a), \theta(z - a + \tau) = \exp(- \pi i \tau - 2\pi i (z-  a)) \theta(z-a)$となる.
よって,
\begin{equation*}
 \prod_{1 \le i \le k} \frac{\theta(z -a_i + 1 )}{\theta(z - b_i + 1)}
 = \prod_{1 \le i \le k} \frac{\theta(z -a_i )}{\theta(z - b_i)}
\end{equation*}
かつ
\begin{align*}
 \prod_{1 \le i \le k} \frac{\theta(z -a_i + \tau )}{\theta(z - b_i + \tau)} &  = \prod_{1 \le i \le k} \frac{\exp(-\pi i \tau - 2\pi i (z-  a_i))\theta(z -a_i )}{\exp(-\pi i \tau - 2\pi i (z-  b_i)) \theta(z - b_i)} \\
  & = \prod_{1 \le i \le k} \frac{\exp(2 \pi i a_i) \theta(z - a_i)}{\exp(2\pi i b_i) \theta(z - b_i)} \\
  & = \exp(2\pi i \sum(a_i - b_i)) \prod_{1 \le i \le k} \frac{\theta(z - a_i)}{\theta(z - b_i)} \\
  & = \prod_{1 \le i \le k} \frac{\theta(z -a_i)}{\theta(z - b_i)}
\end{align*}
となる.よって$\Lambda_{\tau}$上周期的な有理型関数になる.

\subsection{Second logarithmic derivatives}
$\log \theta(z + 1) = \log \theta(z), \log \theta(z+ \tau) = \log \theta(z) - (\pi i \tau - 2 \pi i z)$となるので,

\begin{equation*}
    \frac{d^2}{dz^2}\log \theta(z + \tau)= \frac{d^2}{dz^2}\log \theta(z)
\end{equation*}
とり,二重周期関数である.
また
\begin{equation*}
    \mathfrak{p}(z) = - \frac{d^2}{dz^2} \log \theta_{1, 1}(z) + const
\end{equation*}
となることを示す.
$\mathfrak{p}$関数の定義を思い出すと以下の形であった.
\begin{equation*}
    \mathfrak{p}(z) = \sum_{(n,m) \neq 0} \frac{1}{(z - n \omega_1 - m \omega_2)^2} - \frac{1}{(n \omega_1 + m \omega_2)^2 } + \frac{1}{z^2}
\end{equation*}
これは二重周期を持ち$\mathbb{Z}\omega_1 + \mathbb{Z}\omega_2$上に二次の極を持つ.

$f$が偶関数であって,$\mathbb{Z}\omega_1 + \mathbb{Z}\omega_2$に一位のゼロ点を持ち,$\mathbb{Z}\omega_1 + \mathbb{Z}\omega_2$以外で零点を持たない二重周期関数の場合,
$f'(\omega_1) , f'(\omega_2) \neq 0$なので,$\frac{d}{dz} \log f = \frac{f'}{f}$は$\omega_1, \omega_2$で一次の極を持つ.また
\begin{equation*}
\frac{d^2}{dz^2}\log f = \frac{ f f'' - f'^2 }{f^2}
\end{equation*}
であり, $\omega \in \mathbb{Z}\omega_1 + \mathbb{Z}\omega_2$に対し,二次の極を持つ.
二次の極のローラン級数展開したときの$z^{-2}$次の係数を求めたい.
$f$は原点の近傍で正則なので,テイラー展開し$f(z)=\sum a_nz^n$とする.$a_0=0, a_1\neq 0$となり,
\begin{equation*}
\frac{... -  (\sum_{n \ge 1} n a_n z^{n-1})^2 }{z^2 (\sum_{n \ge 1} a_nz^{n-1})^2}
\end{equation*}
...は定数項がないので,$z^{-2}$の係数は-1となる.他 $\omega \in \mathbb{Z}\omega_1 + \mathbb{Z}\omega_2$についても同様.
$f/f'$もmeromorphicなので,$\frac{d^2}{dz^2}\log f$の-1乗の係数は0になる.
よって$\frac{d^2}{dz^2}\log f + \mathfrak{p}$は全域で極を持たない二重周期関数なので,定数関数となることがわかる.
$\theta_{1,1}$の定理4・1から$\mathbb{Z}+ \mathbb{Z}\tau$上一位のゼロ点のみを持つので$f$の条件を満たし,上の関係式を得る.


\subsection{Sums of first logarithmic derivatives}
$a_i \in \mathbb{C}, \lambda_i \in \mathbb{C}$で$\sum \lambda_i = 0$とする.

この時

\begin{equation*}
    \frac{d}{dz}\log \theta(z + \tau)= \frac{d}{dz}\log \theta(z) - 2\pi i
\end{equation*}
となるので,
\begin{align*}
    \sum \lambda_i \frac{d}{dz}\log \theta(z- a_i + \tau)   &= \sum \lambda_i (\frac{d}{dz}\log \theta(z- a_i)  - 2 \pi i) \\
     & = \sum \lambda_i \frac{d}{dz}\log \theta(z- a_i) \\
\end{align*}
となるので,周期的な関数である.



1番目と2番目を関係を見る.$\theta_{ab}(2z)$を$\theta$の積で表す.
例えば
前回やった関係式R18$(\theta_{ij}^x = \theta_{i, j}(x)$と同じにより,
\begin{align*}
    \theta_{00}^x \theta_{01}^y \theta_{10}^u \theta_{11}^v + \theta_{01}^x \theta_{00}^y \theta_{11}^u \theta_{10}^v
   + \theta_{10}^x \theta_{11}^y \theta_{00}^u \theta_{01}^v + \theta_{11}^x \theta_{10}^y \theta_{01}^u \theta_{00}^v =
   2\theta_{11}^{x_1}\theta_{10}^{y_1}\theta_{01}^{u_1}\theta_{00}^{v_1}
\end{align*}

ただし,$x_1$等は以下で定めている.
\begin{align*}
    x_1 = \frac{1}{2}(x + y + u+v), y_1 = \frac{1}{2}(x + y -u -v), u_1 = \frac{1}{2}(x -y + u-v), v_1 = \frac{1}{2}(x -y -u + v)
\end{align*}
に
$x =  y = u =v =z$とすると,
$x_1 = 2z, y_1 =0, u_1 = 0, v_1 = 0$より
\begin{align*}
    2 \theta_{11}(2z)\theta_{10}(0) \theta_{01}(0) \theta_{00}(0) = 4 \theta_{00}(z)\theta_{01}(z)\theta_{10}(z)\theta_{11}(z) \\
\end{align*}
となる.
すいません.$\theta_{00}(2z)$を積で表すのが難しくて...


2番目と3番目の関係を見る.
またA10
\begin{align*}
   \theta_{11} (x + u) \theta_{11}(x-u)\theta_{00}^2(0) = \theta_{11}^2(x)\theta_{00}^2(u) - \theta_{00}^2(x)\theta_{11}^2(u)
\end{align*}
に対し$u$で2回微分すると,
\begin{align*}
   (\theta_{11} (x + u) \theta_{11}(x-u)\theta_{00}^2(0))'' & = ((\theta_{11}'(x+u)\theta_{11}(x-u)\theta_{00}^2(0) - (\theta_{11}(x+u)\theta_{11}'(x-u)\theta_{00}^2(0) )' \\
    & = \theta_{11}''(x+u)\theta_{11}(x-u)\theta_{00}^2(0) - 2(\theta_{11}'(x+u)\theta_{11}'(x-u)\theta_{00}^2(0))  + \theta_{11}(x+u)\theta_{11}''(x-u)\theta_{00}^2(0) \\
    & = 2\theta_{11}^2(x)(\theta_{00}'^2(u) + \theta_{00}(u)\theta_{00}''(u)) -2\theta_{00}^2(x)(\theta_{11}'^2(u) + \theta_{11}(u)\theta_{11}''(u))
\end{align*}
これに$x=z, u=0$を代入する.
\begin{align*}
2\theta_{11}''(z)\theta_{11}(z)\theta_{00}^2(0) - 2\theta_{11}'(z)^2\theta_{00}^2(0)  = 2\theta_{11}^2(z) (\theta_{00}(0) \theta_{00}''(0) )  - 2\theta_{00}(z)^2 \theta_{11}'(0)^2  \\
\end{align*}
となる.ただし,$\theta_{00}'(0) = 0, \theta_{11}''(0) = 0$を使った.

\begin{align*}
    \frac{d^2}{dz^2} \log \theta_{11} = \frac{\theta_{11}\theta_{11}'' - \theta_{11}'^2}{ \theta_{11}^2} \\
    =\frac{\theta_{11}^2(z) (\theta_{00}(0) \theta_{00}''(0) )  - \theta_{00}(z)^2 \theta_{11}'(0)^2 }{\theta_{00}^2(0) \theta_{11}^2} \\
    = \frac{\theta_{00}''(0)}{\theta_{00}(0)} - \frac{\theta_{00}(z)^2 \theta_{11}'(0)^2 }{\theta_{00}^2(0) \theta_{11}^2}
\end{align*}

$\mathfrak{p}$の微分方程式を最後に導いているが,これはローラン級数展開して係数を調整して,正則な二重周期なので、定数という関係を使う.
つまり原点の近傍で$\mathfrak{p}(z) = \frac{1}{z^2} + az^2 + bz^4 + ...$と表わせ,
\begin{equation*}
    \mathfrak{p}'(z) = - \frac{2}{z^3} + 2az + 4bz^3...
\end{equation*}
より,$\mathfrak{p}'(z)^2 - 4 \mathfrak{p}(z)^3 + 20 a \mathfrak{p}(z) = const$となる.

\section{The functional equation of $\theta(z, \tau)$}

ここでは$\theta$関数の$\tau$を動かすことにより$\tau$の間の関係式を作る.
基本的な方針としては
\begin{itemize}
    \item $\Psi(y, \tau)$という関数を$\theta$に適当にかけて作り,これが$\theta(y, \frac{a \tau + b}{c \tau + d})$の定数倍(実際には$\tau$の関数)になっていることを示す.
    \item 上の$\tau$の関数$\phi(\tau)$を丁寧に計算して求める.
\end{itemize}
の2つである.
\begin{equation*}
\begin{pmatrix}
a & b \\
c & d \\
\end{pmatrix} \in \mathrm{SL}_2(\mathbb{Z})
\end{equation*}
ab, cd are even and $c \ge 0$とする

\begin{equation*}
\Psi(y, \tau) = \exp(\pi i c (c \tau + d)y^2) \theta((c\tau + d)y, \tau)
\end{equation*}
とすると,$\Psi(y + 1, \tau) = \Psi(y, \tau)$となる.
それは
$\frac{\Psi(y+1, \tau)}{\Psi(y, \tau)} = 1$を示せばよく,
\begin{align*}
\theta((c\tau +d)(y+1), \tau) / \theta((c \tau + d)y, \tau)
= \exp(\pi i c (c \tau + d)y^2) /\exp(\pi i c ( c \tau + d ) (y+1)^2) = \exp(- \pi i c (c \tau + d)(2y + 1))
\end{align*}
を示せば良い。
また,
\begin{align*}
\theta((c \tau + d)(y+1), \tau) & = \sum \exp(\pi in^2 \tau + 2\pi in (c \tau +d) y + 2\pi in (c \tau +d)) & \exp(2\pi ind) = 1 \\
& = \sum \exp(\pi i (n + c)^2 \tau- c^2 \pi i \tau + 2\pi in (c \tau +d) y )  & \mbox{ただの展開} \\
& = \sum \exp(\pi i (n )^2 \tau- c^2 \pi i \tau + 2\pi i(n-c) (c \tau +d) y )  & n\mbox{の置換} \\
& = \sum \exp(\pi i (n )^2 \tau +2\pi in (c \tau +d) y  - c^2 \pi i \tau - 2\pi iy (c \tau +d) (c) )  \\
& = \theta( (c \tau + d)y, \tau) \exp( - c^2 \pi i \tau - 2\pi iy (c \tau +d) (c) )  \\
& = \theta( (c \tau + d)y, \tau) \exp( -  \pi i  c ( c\tau  + d) - 2\pi iy (c \tau +d) (c) )  & \exp(-\pi icd) = 1
\end{align*}
より、確認できる.

また、$\tau$を含めた関係式として以下を示す。
\begin{equation*}
 \Psi(y+ \frac{a \tau + b}{c \tau + d}, \tau)  = \exp(-\pi i \frac{a \tau + b}{c \tau + d} - 2\pi iy)\Psi(y, \tau)
\end{equation*}

これは,
\begin{align*}
 \Psi \left(y + \frac{a \tau +b}{c \tau + d}, \tau \right)  & = \exp \left(\pi i c ( c \tau + d) \left(y + \frac{a \tau +b}{c \tau + d}\right)^2 \right)  \theta( (c \tau + d)y + a \tau + b, \tau) \\
 & = \exp \left( \pi i c ( c \tau + d) y^2  + 2\pi i cy (a \tau + b ) + \pi ic \frac{(a \tau + b)^2}{ c \tau + d} \right)\theta( (c \tau + d)y + a \tau + b, \tau)
\end{align*}
となる.
また,$\theta(z + a\tau + b, \tau) = \exp(-\pi i a^2 \tau - 2 \pi i az)\theta(z,\tau)$なことに注意すると,
\begin{align*}
\frac{\theta( (c \tau + d)y  + a \tau + b, \tau)}{\Psi(y, \tau)} & =
\frac{\exp(-\pi i a^2 \tau - 2 \pi i a (c \tau + d)y)\theta( (c \tau + d)y,\tau)}{\exp(\pi i c (c \tau + d)y^2) \theta((c\tau + d)y, \tau)} \\
 & = \exp( - \pi i a^2 \tau - 2 \pi i a (c \tau + d)y  - \pi i c (c \tau + d)y^2)
\end{align*}
となる。これを上の式と合わせることで

\begin{align*}
\frac{\Psi \left(y + \frac{a \tau +b}{c \tau + d}, \tau \right)}{\Psi(y, \tau)} &= \exp(  \pi i c ( c \tau + d) y^2  + 2\pi i cy (a \tau + b ) + \pi ic \frac{(a \tau + b)^2}{c \tau + d} + (- \pi i a^2 \tau - 2 \pi i a (c \tau + d)y  - \pi i c (c \tau + d)y^2)) \\
  & =  \exp( -2\pi i y(ad -bc) + \pi ic \frac{(a \tau + b)^2}{c \tau + d}  - \pi i a^2 \tau ) \\
  & = \exp(-2 \pi i y -  \frac{\pi i }{c \tau + d}(a^2 \tau (c \tau + d) - c (a\tau + b)^2) \\
  & = \exp(-2\pi i y -  \frac{\pi i }{c \tau + d} (a^2 \tau d - 2abc \tau  - b^2c))
\end{align*}
となる.

\begin{align*}
a^2 \tau d - 2abc \tau  - b^2c & =  a(ad -bc)\tau -ab(c \tau  +d) + b(ad -bc) \\
                               & =  a\tau + b - ab( c\tau + d)
\end{align*}
$ab$が偶数なので,$\exp(-2\pi i) =1$に注意すると,
\begin{equation*}
 \exp(-2\pi i y -  \frac{\pi i }{c \tau + d} (a^2 \tau d - 2abc \tau  - b^2c)) = \exp(-2\pi iy  -\pi i \frac{a \tau + b}{c \tau + d})
\end{equation*}
となる


以前の章で求めた$\theta$関数の一意性から$f(z + \tau)=\exp(az + b)f(z), f(z+1) = f(z)$と表される正則関数関数は
$f(z) = \alpha \theta(z, \tau)$となる.
$\tau' := \frac{a \tau + b}{c \tau + d}$とすると,
この条件から$\Psi(y, \tau') = \phi(\tau) \theta(z, \tau')$
と表される.

$\sum \exp(\pi i n^2 \tau + 2\pi in z)$よりフーリエ係数を見ることで$\theta(z, \tau)$の0次成分は1となる.
よって,
\begin{equation*}
\int_0^1 \theta(y, \tau) dy = 1
\end{equation*}
となる.
よって,$\Psi(y, \tau') = \phi(\tau) \theta(y, \tau')$を積分すると,
\begin{equation*}
    \int_0^1 \phi(\tau) \theta(y, \tau') dy = \phi(\tau) = \int_0^1 \Psi(y, \tau')dy = \int_0^1 \exp(\pi ic (c \tau +d)y^2) \theta((c \tau +d)y, \tau)dy
\end{equation*}
$c =0$の時,

\begin{align*}
\int_0^1 \exp(\pi ic (c \tau +d)y^2) \theta((c \tau +d)y, \tau)dy
& =  \int_0^1 \theta(dy, \tau)dy \\
& =  \int_0^{d} \theta(z, \tau)dz 1/d  & z = dy \mbox{で置換}\\
& =  \int_0^{1} \theta(z, \tau)dz d  & z = dy d = \pm 1 \mbox{で}, \theta(-z, \tau) = \theta(z, \tau)より\\
& = d
\end{align*}
となる. $c > 0$を考える.
\begin{align*}
\phi(\tau)
& = \int_0^1 \exp(\pi i c  (c\tau + d)y^2)(\sum \exp(\pi i n^2 \tau  + 2\pi in y(c \tau + d))) \\
& = \int_0^1 \sum \exp(\pi i (\tau + \frac{d}{c}) (c^2y^2 + 2cny  + n^2)) - \pi i((\tau + \frac{d}{c})n^2 +  n^2 \tau)) \\
& = \sum \exp(-\pi i n^2 \frac{d}{c}) \int_0^1\exp(\pi i (\tau + \frac{d}{c})(cy + n)^2) dy
\end{align*}
となる.
$cd$がevenなことに注意すると等式の右辺最初の項は
\begin{equation*}
\exp(- \pi i d (n+c)^2 / c) = \exp(-\pi i n^2 \frac{d}{c} - 2\pi ind - \pi i cd) = \exp(-\pi i n^2 \frac{d}{c})
\end{equation*}
となる.これと
\begin{equation*}
\int_0^1 \exp(\pi (cy + c + n)^2 (\tau  + \frac{d}{c})) dy
= \int_1^2 \exp(\pi (cy  + n)^2 (\tau  + \frac{d}{c})) dy
\end{equation*}
より,
\begin{equation*}
 \phi(\tau)  = \sum_{1\le n \le c} \exp(-\pi i n^2 d/c)\int_{- \infty}^{\infty}\exp(\pi i c^2y^2)(\tau + \frac{d}{c})dy
\end{equation*}
となる.
この積分を積分が簡単に計算できる所で評価し、その後解析接続で広げる.

$\tau = it - \frac{d}{c}$とする.この時,
\begin{align*}
\int_{- \infty}^{\infty}\exp(\pi i c^2y^2)(\tau + \frac{d}{c})dy
& = \int_{- \infty}^{\infty}\exp(- \pi c^2 y^2 t)dy \\
& = \int_{- \infty}^{\infty}\frac{1}{c t^{1/2}}\exp(- \pi u^2)du & u = c t^{1/2}y \mbox{に置換} \\
& = \frac{1}{c t^{1/2}}
\end{align*}
となる.これは$t > 0$の全てで成り立つので,$t= (\tau + \frac{d}{c}) / i$から
\begin{equation*}
\int_{- \infty}^{\infty}\exp(\pi i c^2y^2)(\tau + \frac{d}{c})dy = \frac{1}{ c \left( (\tau + d/c) / i\right)^{1/2}}
\end{equation*}
(ただし1/2乗は実部が正の方を取る.)

\begin{equation*}
 S_{d,c}: = \sum_{1\le n \le c} \exp(-\pi i n^2 d/c)
\end{equation*}
がわかればよい.これはいわゆるガウス和である.
実際ガウス和は$c^{1/2} \zeta$($\zeta$は1の8乗根)となる.ここでは
その全ては証明しないが,$d=2, c$が奇素数$p$の時だけ示す.
その前にJacobi Symbolについて定義しておく.

Jacobi記号は
\begin{equation*}
\left(\frac{a}{p}\right)=\left\{\begin{aligned} 0 & \text { if } a \equiv 0(\bmod p) \\ 1 & \text { if } a \neq 0(\bmod p) \text { and for some integer } x: a \equiv x^{2}(\bmod p) \\-1 & \text { if } a \neq 0(\bmod p) \text { and there is no such } x \end{aligned}\right.
\end{equation*}
で分母が素数の場合を定義し
$n=p_{1}^{\alpha_{1}} p_{2}^{\alpha_{2}} \cdots p_{k}^{\alpha_{k}}$の時,
\begin{equation*}
\left(\frac{a}{n}\right)=\left(\frac{a}{p_{1}}\right)^{\alpha_{1}}\left(\frac{a}{p_{2}}\right)^{\alpha_{2}} \cdots\left(\frac{a}{p_{k}}\right)^{\alpha_{k}}
\end{equation*}
で定めたものである.
分母が1の場合は必ず1とする

Jacobi記号に関する定理として以下は使う
\begin{lem}
\begin{itemize}
    \item $\left( \frac{a}{n} \right) \left( \frac{b}{n}\right) = \left(  \frac{ab}{n}\right)$
    \item $m,n$が互いに素な奇数とするこの時
    \begin{equation*}
       \left(\frac{m}{n}\right)\left(\frac{n}{m}\right)=(-1)^{\frac{m-1}{2} \cdot \frac{n-1}{2}}
    \end{equation*}
    \item 
    \begin{equation*}
    \left(\frac{a}{n}\right)=\left(\frac{a \pm m \cdot n}{n}\right)
    \end{equation*}
    \item 
    \begin{equation*}
    \left(\frac{2}{n}\right)=(-1)^{\frac{n^{2} - 1}{8}}
    \end{equation*}
\end{itemize}
\end{lem}

実際にガウス和を計算する.$1 \le n, m \le p$に対し,
$n^2 - m^2 = (n-m)(n+m)$が$p$で割り切れるのは$n+m$が$p$で割り切れる時である.
この時$(p-k)^2 \equiv k^2 \quad \mathrm{mod} p$となり,
$\exp(2 \pi i k^2/p) = \exp(2 \pi i (p-k)^2/p)$となる.
それ以外の組み合わせでは一致しないので,
$\sum \exp( 2\pi i k^2/ p)$は
$0$を除き$n \equiv x^2$ mod $p$となる$n$がちょうど2回ずつ現れる.
以下では$ \exp( 2\pi i / p) = \zeta_p$と表す.
\begin{equation*}
\sum_{n=0}^{p-1}  \zeta_p^n = (1 - \zeta_p^p) / (1 -\zeta_p) = 0
\end{equation*}
となることに注意し、
$n=0$も含め,$n \equiv x^2$ mod $p$となる$n$全てを走る和は
$n \equiv x^2$ mod $p$とかけない$n$全てを走る和の-1倍に一致する.
よって,
\begin{equation*}
\sum_{n=0}^{p-1} \exp(2\pi i n^2/p) = \sum_{n =1}^{p-1}  \left(\frac{n}{p}\right)\exp(2 \pi i n/p)
\end{equation*}
となる.

後は
\begin{equation*}
\tau_{p}=\sum_{a=1}^{p-1}\left(\frac{a}{p}\right) \zeta_{p}^{a}
\end{equation*}
とした時,
$\tau_{p}^{2}=(-1)^{\frac{p-1}{2}} p$となることを示す.
\begin{align*}
\tau_{p}^2=\sum_{a=1}^{p-1} \sum_{b=1}^{p-1}\left(\frac{ab}{p}\right) \zeta_{p}^{a+b}
\end{align*}
となり,$b = at \equiv \mathrm{mod} p$となる
$t$の代表元として$1, \ldots, p-1$までが取れるので
$\left( \frac{a^2}{p} \right) =1$に注意すると
\begin{equation*}
\tau_{p}^{2}=\sum_{a=1}^{p-1} \sum_{t=1}^{p-1}\left(\frac{a^{2} t}{p}\right) \zeta_{p}^{a+a t}=\sum_{t=1}^{p-1}\left(\frac{t}{p}\right)
\sum_{a=1}^{p-1} \zeta_{p}^{a(t+1)}
\end{equation*}
となる.
$t+ 1 =p$以外では
$\sum_{a=1}^{p-1} \zeta_{p}^{a(t+1)} = -1$となり,$t+1=p$の時は$p-1$となるので,
\begin{equation}
\tau_{p}^{2}=-\sum_{t=1}^{p-1}\left(\frac{t}{p}\right)+\left(\frac{p-1}{p}\right) p=\left(\frac{-1}{p}\right) p
\end{equation}
となる.
ただし$\sum_{t=1}^{p-1}\left(\frac{t}{p}\right)  = 0$に注意.
これは$a$を$p$を法として平方剰余を持たない$a$を使い,

\begin{equation*}
\sum_{t=1}^{p-1}\left(\frac{t}{p}\right) = 
\left( \frac{a}{p}\right)
\sum_{t=1}^{p-1}\left(\frac{t}{p}\right)
\end{equation*}
となるので,0とわかる.


これで
\begin{equation*}
\Psi(y, \tau) = \exp(\pi i c (c\tau + d ) y^2) \theta( (c \tau + d)y, \tau) = \zeta c^{1/2} \frac{1}{ c ((\tau + \frac{d}{c} )/ i)^{1/2}} \theta(y, \frac{a \tau + b}{ c \tau + d})
\end{equation*}
これをもとに$y = \frac{z}{c \tau + d}$を代入した結果以下が得られる.
\begin{equation*}
\theta( \frac{z}{c \tau + d}, \frac{a \tau + b}{c \tau + d}) = \zeta^{-1}  ((c \tau + d)/i)^{1/2} \exp(\pi i c z^2/ (c \tau + d)) \theta(z, \tau)
\end{equation*}

ここから1のべき乗根をまとめてその値を求める.
\begin{thm}
$a,b,c,d \in \mathbb{Z}$で$ad -bc=1,ab, cd:even$とする.
この時,
\begin{align*}
 \theta(\frac{z}{c\tau + d}, \frac{a \tau + b}{c \tau +d}) =  \zeta
 \left(c\tau + d \right)^{1/2}
 \exp(\pi i c z^2/(c\tau + d))  \theta(z, \tau)
 \tag{F1}
\end{align*}
となる
今$c > 0$ または$c=0$ and $d>0$とする.
この時
\begin{enumerate}
    \item $c$がeven,$d$がoddなら,
    \begin{equation*}
     \zeta = i^{\frac{1}{2}(d-1)}  \left(\frac{c}{|d|} \right)
    \end{equation*}
    ただし($\frac{x}{y}$)はJacobi Symbol
\item もし$c$がodd, $d$がevenの時,
\begin{equation*}
 \zeta = \exp(-\pi i c /4)  \left(\frac{d}{c} \right)
\end{equation*}
\end{enumerate}

\end{thm}
\begin{proof}
inductionで示す.
最初は$
\begin{pmatrix}
a & b \\ c & d
\end{pmatrix}
= \begin{pmatrix}
1 & b \\ 0 & 1
\end{pmatrix}
,b$はevenの場合を考える.
この時,
\begin{align*}
\theta(z, \tau + b)
& = \sum \exp(\pi i n^2 (\tau + b) + 2\pi inz) \\
& = \sum \exp(\pi i n^2 (\tau ) + 2\pi inz)  = \theta(z, \tau)
\end{align*}
とり$F1$の右辺も$c =0$に注意すると,
$\zeta  \theta(z, \tau)$となり,
今は$\zeta =1 = i^0 1$となるので成り立つ.
$\begin{pmatrix}
a & b \\ c & d
\end{pmatrix}
= \begin{pmatrix}
0 & -1 \\ 1 & 0
\end{pmatrix}
$の時,
(F1)から
\begin{equation*}
    \theta(z/\tau, -1/\tau) = \zeta t^{1/2} \exp(\pi i z^2/\tau) \theta(z, \tau)
\end{equation*}
となる.
また,
\begin{equation*}
 \theta(z, \tau)  = \sum_{1\le n \le c}\exp(-\pi in^2 \frac{d}{c}) \frac{1}{ c ((\tau + \frac{d}{c})/i)^{1/2}} \exp(-\pi i c z^2/ (c \tau + d)) \theta(\frac{z}{c \tau + d}, \frac{a \tau + b}{ c\tau + d})
\end{equation*}
となっており,これに今回の条件を代入すると
\begin{align*}
\theta(\frac{z}{\tau}, \frac{-1}{\tau})  & = \sum_{1 \le n \le 1} (\tau / i)^{1/2} \exp(\pi i \frac{z^2}{\tau}) \theta(z , \tau) \\
& = \exp(- \pi i /4) \tau^{1/2} \exp(\pi i z^2 / \tau) \theta(z, \tau)
\end{align*}
となる.これから成り立つ.
この後は|c| + |d|のinductionから成り立つことが言える.
このinductionは式変形とJacobi Symbolの真面目な計算で求められる.

$|d| > |c|$とする.この時
$|d-2c|$か $|d + 2c|$いずれかは $|d|$より小さくなる.
今仮に $|d + 2c| < |d|$とする.この時$|d+2c| + |c|$についてはinductionの仮定を満たすとして、そこに帰着させる形で証明する.

$F1$に$\tau$として$\tau + 2$を取り,
$
\begin{pmatrix}
a & b  \\ c  & d 
\end{pmatrix}
\in \mathrm{SL}(2, \mathbb{Z})
$
の場合の等式を見ると
\begin{equation*}
\theta(\frac{z}{c \tau + 2c + d}, \frac{a\tau + 2a + b}{c 
\tau  + 2c + d}) = \zeta (c \tau +2c + d)^{1/2}\exp(\frac{\pi i c z^2}{c \tau + 2c + d})\theta(z, \tau + 2)
\end{equation*}
となる.

$
\begin{pmatrix}
a & b + 2a \\ c  & d + 2c 
\end{pmatrix}
\in \mathrm{SL}(2, \mathbb{Z})
$の場合はの$F1$は
\begin{equation*}
\theta(\frac{z}{c \tau + 2c + d}, \frac{a\tau + 2a + b}{c 
\tau  + 2c + d}) = \zeta (c \tau +2c + d)^{1/2}\exp(\frac{\pi i c z^2}{c \tau + 2c + d})\theta(z, \tau )
\end{equation*}
となる.
$\theta(z, \tau) = \theta(z, \tau + 2)$
となることから,
後者の$\zeta$を$\zeta'$と表すと
$\zeta = \zeta'$となるので,$\zeta'$が$\zeta$の条件を満たす
つまり
\begin{itemize}
    \item  $c$がoddで$d$がevenの時$\exp(- \pi i c /4) \left(\frac{d + 2c}{c}\right) = \exp(- \pi i c /4) \left(\frac{d}{c} \right)$
    \item $c$がevenで$d$がoddの時に
\begin{equation*}
  i^{\frac{1}{2}(d -1) }\left(\frac{c}{|d|} \right) = i^{\frac{1}{2}(d + 2c -1) }\left(\frac{c}{|d + 2c|}\right) 
\end{equation*}
\end{itemize}
を示せば良い.

c がodd, dがevenの場合
Jacobi Symbolの演算から
$(\frac{d + 2c}{c}) = (\frac{d}{c})$となるので一致する.
cがeven, dがoddの場合に示す.
これを示すために一つ一つ計算していく.
$c = 2^kc'$($c'$はodd)とする.

\begin{align*}
\left(\frac{c}{|d|} \right) 
& = \left(\frac{2}{|d|} \right)^k \left(\frac{c'}{|d|}\right) \\
& = (-1)^{\frac{|d|^2 -1}{8}k}\left(\frac{c'}{|d|}\right)
\end{align*}
また,
\begin{equation*}
\left( \frac{c'}{|d|} \right) \left( \frac{|d|}{c'} \right)
= (-1)^{\frac{|d|-1}{2} \frac{c'-1}{2}}
\end{equation*}
となる.
同様に
\begin{align*}
\left(\frac{c}{|d + 2c|} \right) 
& = \left(\frac{2}{|d + 2c|} \right)^k \left(\frac{c'}{|d + 2c|}\right) \\
& = (-1)^{\frac{|d + 2c|^2 -1}{8}k}\left(\frac{c'}{|d+2c|}\right)
\end{align*}
\begin{equation*}
\left( \frac{c'}{|d + 2c|} \right) \left( \frac{|d + 2c|}{c'} \right)
= (-1)^{\frac{|d + 2c|-1}{2} \frac{c'-1}{2}}
= (-1)^{\frac{|d |-1}{2} \frac{c'-1}{2}}
\end{equation*}
となる.($c$はevenなので)
\begin{equation*}
    \left(\frac{|d + 2c| }{c'} \right) = \left( \frac{|d|}{c'}\right)
\end{equation*}
より,
\begin{equation*}
\left( \frac{c'}{|d + 2c|}\right) =  \left( \frac{c'}{|d|} \right)
\end{equation*}
となる.
よって
\begin{align*}
\left( \frac{c}{|d + 2c|} \right) 
& = (-1)^{\frac{|d + 2c|^2 -1}{8}k}\left( \frac{c'}{|d + 2c|}\right)  \\
& = (-1)^{\frac{|d + 2c|^2 -1}{8}k} \left( \frac{c'}{|d|} \right) \\
& = (-1)^{(\frac{|d + 2c|^2 -1}{8} - \frac{|d|^2}{8}) k} \left( \frac{c}{|d|} \right) \\
\end{align*}
となる.
\begin{equation*}
(-1)^{(\frac{|d + 2c|^2 }{8} - \frac{|d|^2}{8}) k} = (-1)^{\frac{4cd + 4c^2}{8}k}
\end{equation*}
となり,これは$k=1$の時-1となり$k >1$の時1となる.
これは$(-1)^{c/2}$と一致する.
$|d-2c| < |d|$の場合も同様.

|c| > |d|の場合は$|d| > |c|$の場合に帰着させる.
\begin{align*}
\theta(\frac{z}{t}, \frac{-1}{\tau})  
& = \exp(- \pi i /4) \tau^{1/2} \exp(\pi i z^2 / \tau) \theta(z, \tau)
\end{align*}
であることに注意して
$\tau$として$- \frac{1}{\tau}$を取り,$w= \tau z$として$F1$を変形させると,
\begin{align*}
\theta \left( \frac{w}{-c + d \tau}, \frac{-a + b \tau}{-c + d \tau} \right) 
& = \zeta  (\frac{-c}{\tau} + d)^{1/2} \exp \left( \pi i c (\tau z)^2/ \tau( -c + d \tau) \right) \theta(\frac{\tau z}{\tau} , \frac{-1}{\tau}) \\
& = \zeta  (\frac{-c}{\tau} + d)^{1/2} \exp \left( \pi i c (w)^2/ \tau( -c + d \tau) \right) \exp(- \pi i /4) \tau^{1/2} \exp(\pi i w^2 / \tau) \theta(w, \tau) \\
& = \zeta  \exp(- \pi i /4) (-c + d \tau)^{1/2} \exp( \pi i c (w)^2/ \tau( -c + d \tau ) + \pi i w^2 / \tau) \theta(w, \tau) \\
& = \zeta  \exp(- \pi i /4) (-c + d \tau)^{1/2} \exp( \pi i d (w)^2/ ( -c + d \tau ) ) \theta(w, \tau)
\end{align*}
($ \frac{c}{-c + d \tau} + 1 =  \frac{d\tau}{ -c + d \tau}$より)
となり$|c| > |d|$の場合の関係式に帰着できる.後は真面目に計算すれば証明される(と思うので省略する).
\end{proof}


\section{The Concept of modular forms}

自然な射$\gamma_N: \mathrm{SL}(2, \mathbb{Z})  \to \mathrm{SL}(2, \mathbb{Z}/N)$と表す.
$\mathrm{Ker}\gamma_N$を$\Gamma_N$と表し,\textbf{level N-principai congruence subgroup}という.
特にlevel2の場合を見る.
$\mathrm{SL}(2, \mathbb{Z}/2)$は
$ad -bc =1$の条件から$ad =0,bc=1$か$ad=1,bc=0$となり,
$ad=1$,つまり$a=d=1$の場合は
\begin{itemize}
    \item $b=c=0$
    \item $b=1, c=0$
    \item $b=0, c=1$
\end{itemize}
の3種類がある.$bc=1$の場合も同様に3種類ある.
列挙しておくと
\begin{equation*}
\begin{pmatrix}
1 & 0 \\
0 & 1
\end{pmatrix},
\begin{pmatrix}
1 & 1 \\
0 & 1
\end{pmatrix},
\begin{pmatrix}
1 & 0 \\
1 & 1
\end{pmatrix},
\begin{pmatrix}
0 & 1 \\
1 & 0
\end{pmatrix},
\begin{pmatrix}
1 & 1 \\
1 & 0
\end{pmatrix},
\begin{pmatrix}
0 & 1 \\
1 & 1
\end{pmatrix}
\end{equation*}
の6種類となる.これは3次置換群$SL_3$と同型になる.

We define, following Igusa, $\Gamma_{1,2} \subset \mathrm{SL}(2,\mathbb{Z})$
to be $\gamma_2^{-1}$ of the sugroup of $\mathrm{SL}(2, \mathbb{Z}/2)$ consisting of
$\begin{pmatrix}
1 & 0 \\
0 & 1
\end{pmatrix},
\begin{pmatrix}
0 & 1 \\
1 & 0
\end{pmatrix}
$.
Clearly this is the subset of $\mathrm{SL}(2, \mathbb{Z})$ of elements
$
\begin{pmatrix}
a & b \\
c & d
\end{pmatrix}
$ such that $ab, cd$ are even.
($\mathrm{SL}(2, \mathbb{Z}/2)$の残りの4つをみれば,わかる)
Note however that whereas $\Gamma_N$ is a normal subgroup of $\mathrm{SL}(2, \mathbb{Z}/2)$, $\Gamma_{1,2}$ is not; it has 2 conjugates
\begin{equation*}
\gamma_2^{-1} \left( \begin{pmatrix}
1 & 0 \\
0 & 1
\end{pmatrix},
\begin{pmatrix}
1 & 1 \\
0 & 1
\end{pmatrix}
\right)
\mbox{ and }
\gamma_2^{-1} \left( \begin{pmatrix}
1 & 0 \\
0 & 1
\end{pmatrix},
\begin{pmatrix}
1 & 0 \\
1 & 1
\end{pmatrix}
\right)
\end{equation*}
実際
$\begin{pmatrix}
1 & 1 \\
0 & 1
\end{pmatrix}$の逆行列が自身なことに注意すると
\begin{equation*}
\begin{pmatrix}
1 & 1 \\
0 & 1
\end{pmatrix}
\begin{pmatrix}
0 & 1 \\
1 & 0
\end{pmatrix}
\begin{pmatrix}
1 & 1 \\
0 & 1
\end{pmatrix}
=
\begin{pmatrix}
1 & 0 \\
1 & 1
\end{pmatrix}
\end{equation*}
となるので,自身とは一致ししない共役部分群が存在することがわかる.

They are the groups for which $\theta_{01}$ and $\theta_{10}$ have functional equations.
これが何を言ってるか意味がよくわからない.(関数等式を持つ$\theta_{01}, \theta_{10}$についての群?とは→これら×elementary facotrになる群)
If  we write out
\begin{equation*}
    \theta(z/(c \tau + d), (a \tau +b)/(c \tau + d))
\end{equation*}
when $
\begin{pmatrix}
a & b \\
c & d
\end{pmatrix}
 \notin \Gamma_{1,2}$.
 we find that it is an elementary factor times $\theta_{01}$ or $\theta_{10}$
 実際に作用を確認するにあたり,$\mathrm{SL}(2, \mathbb{Z})$のgenerator
 $
\begin{pmatrix}
1 & 1 \\
0 & 1
\end{pmatrix},
\begin{pmatrix}
0 & -1 \\
1 & 0
\end{pmatrix}
$での作用を見る.
($\mathrm{SL}(2, \mathbb{Z})$)がこの2つで生成されることは一旦認める

\begin{align*}
\theta_{00}(z, \tau + 1)
& = \sum \exp(\pi i n^2 (\tau + 1) + 2 \pi i n z) \\
& = \sum \exp(\pi i n^2 \tau  + \pi i n + 2\pi i nz) & (n^2-n) \mbox{は偶数なので} \\
& = \theta(z + \frac{1}{2}, \tau) = \theta_{01}(z, \tau) \\
\end{align*}
同様に
\begin{align*}
\theta_{01}(z, \tau + 1)
& = \sum \exp(\pi i n^2 (\tau + 1) + 2 \pi i n z + \pi in ) \\
& = \sum \exp(\pi i n^2 \tau   2\pi i z ) & (n^2+n) \mbox{は偶数なので} \\
& = \theta_{00}(z, \tau)
\end{align*}
また,
\begin{align*}
\theta_{10}(z, \tau + 1)
& = \exp(\pi i (\tau + 1)/4 )\theta(z + \frac{1}{2}(\tau + 1), \tau+1) \\
& = \exp(\pi i (\tau + 1)/4 )\sum \exp(\pi i n^2 (\tau + 1) + 2 \pi i n (z + \frac{1}{2}\tau) + \pi in ) \\
& = \exp(\pi i /4)\theta_{10}(z, \tau)
\end{align*}
同様に
\begin{align*}
\theta_{11}(z, \tau+1) = \exp(\pi i /4) \theta_{11}(z, \tau)
\end{align*}
となる.


\end{document}
